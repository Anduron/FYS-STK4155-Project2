%================================================================
\section{Introduction}\label{sec:Introduction}
%================================================================
In this project, we will apply several statistical learning methods to data from both the one-dimensional and two-dimensional Ising model. The Ising model is a binary value system, and is in statistical mechanics used to model interacting dipole moments of atomic spins that are arranged in a lattice with $L$ spin sites. In the one-dimensional case, the linear regression methods OLS, Ridge and Lasso will be used to determine the coupling constant for the Ising model Hamiltonian. A multilayer neural network will also applied on the regression problem. The two-dimensional Ising model exhibits a phase transition from a magnetic phase to a phase with zero magnetization at a given critical temperature, called the Curie temperature $T_C$. Both logistic regression and neural networks will be used in order to classify the phase of the two-dimensional Ising model. 

In the linear regression problem, we assume no prior knowledge about the origin of the data set. Hence, a Ising model with pairwise interactions between every pair of variables will be used. How this generalization plays out with the linear regression methods will be studied, and the performance metrics MSE and $R^2$ score will be used to assess the models. Learning the Ising Hamiltonian will be done with both linear regression and a neural network.

The goal with the data from the two-dimensional Ising model, which exhibits a phase transition from an ordered phase to a disordered phase at the Curie temperature, is to build a model which will take in a spin configuration and predict whether it constitutes an ordered or disordered phase. We will here use a dataset provided by Metha et. al \cite{Mehta_2019}, which consists of ordered, disordered and "critical-like" phases. The ordered and disordered states will be used to train the logistic regressor and neural network. The remaining critical states will be used as a held-out validation set which the hope is that we will be able to make good extrapolated predictions on. If successful, this may be a viable method to locate the position of the critical point in other complicated models with no known exact analytical solution. 

This project is structured by first presenting a theoretical overview of the Ising model and the aforementioned statistical learning methods in \autoref{sec:Theory}. This is followed by a presentation on the approach to study the various computations of interest in \autoref{sec:Method}. Next, the results of the implementation are presented and discussed in \autoref{sec:Results}, before subsequently they are concluded upon in \autoref{sec:Conclusion}. Lastly, an outline of possible continuations of the models, with respect to the implementation, are presented in \autoref{sec:Future}.