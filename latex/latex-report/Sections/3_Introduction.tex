%================================================================
\section{Introduction}\label{sec:Introduction}
%================================================================
We will determine, using first various regression methods, the value of the coupling constant for the energy of the one-dimensional Ising model. Thereafter, we will use the two-dimensional data, but now computed at different temperatures, in order to classify the phase of the Ising model. Below the critical temperature, the system will be in a so-called ferromagnetic phase. Close to the critical temperature, the final magnetization becomes smaller and smaller in absolute value while above the critical temperature, the net magnetization is zero. This classification case, that is the two-dimensional Ising model, will be studied using logistic regression and deep neural networks.


In statistical mechanics, the Ising model is a model of interacting magnetic dipole moments of atomic spins. The spins are arranged in a lattice, allowing each spin to interact with its neighbors. The two dimensional square lattice Ising model exhibits a phase transition from a magnetic phase (a system with finite magnetic moment) to a phase with zero magnetization at a given critical temperature, called the Curie temperature. (FYS3150, Project 4)


Useful notebooks from \cite{Mehta_2019} \url{https://physics.bu.edu/~pankajm/ML-Notebooks/HTML/}