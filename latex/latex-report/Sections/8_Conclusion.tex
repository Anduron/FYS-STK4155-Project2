%================================================================
\section{Conclusion}\label{sec:Conclusion}
%================================================================

In this project, we have applied several statistical learning methods to data from both the one-dimensional and two-dimensional Ising model. In the one-dimensional case, the linear regression methods OLS, Ridge and Lasso were used to determine the coupling constant for the Ising model Hamiltonian. A multilayer neural network was also applied on the regression problem. The two-dimensional Ising model, on the other hand, exhibits a phase transition from a magnetic phase to a phase with zero magnetization at a given critical temperature, called the Curie temperature $T_C$. Both logistic regression and neural networks was used in order to classify the phase of the two-dimensional Ising model. 

In the linear regression problem, we assumed no prior knowledge about the origin of the data set. Hence, a Ising model with pairwise interactions between every pair of variables was used. However, this generalized model gives rise to that OLS and Ridge regression learn nearly symmetric weights $J=-0.5$. Lasso regression, however, tends to break this symmetry and perfectly determine the coupling $J=-1$ with proper regularization. With regularization $\lambda = 10^{-2}$, Lasso gave an $R^2$ score of 1. 

The two-dimensional Ising model exhibits a phase transition from an ordered phase to a disordered phase at the Curie temperature. The goal was to build a model which will take in a spin configuration and predict whether it constitutes an ordered or disordered phase. The provided dataset consisted of ordered, disordered and "critical-like" phases. Both ordered and disordered states were used to train the logistic regressor. The remaining critical states was used as a held-out validation set which the hope was that we would be able to make good extrapolated predictions on. If successful, this may have been a viable method to locate the position of the critical point in other complicated models with no known exact analytical solution. However, logistic regression was not able to correctly fit the Ising model as it is not a linear model, and was, in the case with both Gradient Descent and Newton-Raphson as optimization methods, no better than just guessing the phase.

Neural networks proved useful to predict the energy of the Ising model (\autoref{fig:NN_reg}), even in the presence of a high number of useless features produced by the generalized Ising model. The final R2-score on test data was $94.3\%$. Regularization was proved crucial in order to eliminate the influence of features that did not contribute to the energy (\autoref{fig:NN_CS}). Compared to OLS and Ridge for determining coupling constants, which is a similar problem, neural networks performed generally better, though not as good as Lasso. 

We managed to reproduce the same order of accuracy when determining the phase of the Ising model as Mehta, for neural networks trained on the ordered/disordered set(\autoref{fig:NN_class}). However, our models did not generalize as good on the critical data set. This may be a shortcoming of our simpler network, using only 400 nodes instead of 1000 as Mehta. This was chosen because of computationally limitations. Another major factor is our very coarse grid search, testing for only 3 learning rates and penalties. 